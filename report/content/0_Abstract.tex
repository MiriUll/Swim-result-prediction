%\begin{itemize}
%	\item Make the readers interested and provide all the necessary information for the readers to decide if the content is relevant to them.
%	\item Summarize the major aspects of your report in a few sentences.
%	\item Describe the problem you are approaching and what solution you have developed.
%	\item Describe the relevance of your work and why your report is a valuable addition to the seminar.
%	\item Do not include lengthy descriptions, abbreviations, figures, unnecessary adverbs, citations or references to other literature.
%	\item Do Not Use Symbols, Special Characters, Footnotes, or Math in Paper Title or Abstract.
%\end{itemize}
Age group swimmers' performance is hard to predict as their performance is highly dependent on their current training status and body development. In this paper, we propose an Android application that helps coaches to estimate the performance of their athletes. Therefore, the coaches can measure the 50m freestyle times in training and use our app to predict the corresponding 100m and 200m freestyle time. We use a linear regression network that predicts these times based on the 50m time as well as the swimmer's age and training age, i.e. how long the swimmer has been training on a competitional level. With this approach, we can predict the 100m time with an error of 1.83s and the 200m time with an error of 7.97 on average. We encounter problems due to non-professional swimmers being unreliable with regards to their competition performance. Moreover, our dataset provides insufficient information about the swimmer and his training status. Nevertheless, we consider our approach as a valuable contribution to help coaches of non-professional junior swimmers to estimate their swimmers' performances.