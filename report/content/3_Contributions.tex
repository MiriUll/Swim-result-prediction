%\begin{itemize}
%	\item Lead up to your contributions.
%	\item Describe your research process.
%	\item Start with theoretical work and work yourself to its applications in your reproach report.
%	\item Document your implementation and solution to the problem described in previous sections.
%	\item Discuss your data collection, training or implementation approach and highlight interesting technical details.
%	\item Feel free to add more sections or subsections and rename existing sections e.g. the Contributions section as you need.
%\end{itemize}
\section{Design and implementation}\label{sec:contribution}
 In contrast to the aforementioned papers, we focus on junior swimmers of any level, not only of national level. In our approach, we try to predict the 100m and 200m freestyle times. As freestyle is the most common strokes in competition, we focus on this stroke for a first evaluation.
\subsection{Data generation}\label{sec:data_gen}
As we could not find a dataset that matches our requirements, we generate a new dataset using competition result published by the DSV (Deutscher Schwimm-Verband e.V.). All swimmers that are eight years or older and take part in competitions in Germany must register at the DSV. The DSV publishes all official competition results as well as information about registered swimmers online \footnote{\url{http://www.dsv.de/schwimmen/wettkampf-national/schwimmerabfrage/}, accessed 02.01.2020}; thus, we can access a considerable amount of data for our dataset.\\
Similar to the records in \cite{Xie.2015} our dataset contains records that look like this: \texttt{(gender, age, training age, 50m time, 100m time, 200m time)}. To get a continuous representation of our data, we express all times in seconds with two decimal places.\\
We include the gender and the age of the swimmer as boys and girls develop differently at different ages. \citet{Golle.2015} showed in their study that the fitness performance improves when children get older. While boys outperformed girls of the same age-group in upper-extremity muscular power and endurance, girls performed better in flexibility test. All of the three aforementioned physical aspects are important for swimmers; thus, we expect different performances by swimmers of different age groups and gender. We use the swimmer's age group to determine his age instead of his actual age at the respective competition.\\
The training age describes how long the swimmer has been training competitive swimming. The longer the swimmer has been training, the better is the swimmer's endurance and his technique. On top of that, he gained experience in competitions; thus, he may be less nervous during competition.\\
\citet{Wakayoshi.1992} have shown that for college swimmers, the distance and the corresponding time have a linear connection thus we can calculate the longer distance time based on any shorter distance. However, \citet{Golle.2015} observed a curvilinear improvement in upper-body strength and endurance for older junior swimmers. Putting this together, we expect the correlation between 50m, 100m, and 200m time to be non-linear, dependent on the age, training age and gender of the swimmer.

\subsection{Machine learning approach}
We use Tensorflow\footnote{\url{https://www.tensorflow.org/} accessed 03.01.2020} for training our machine learning models. Tensorflow offers a well-documented and flexible ecosystem as well a the tensorflow-lite package that makes it possible to use trained models on mobile devices.\\
We will train two separate models, one for the 100m and one for the 200m prediction. Our models must return continuous output values; thus, we need to solve a regression instead of a classification task. Therefore, we use the regression model explained in the Tensorflow tutorials \footnote{\url{https://www.tensorflow.org/tutorials/keras/regression}, accessed 02.01.2020}.
For the training itself, we randomly split our training data into a training and validation subset with a fraction of $\frac{1}{3}$ and use the validation set to evaluate the current model. In order to find the best model, we vary these parameters:
\begin{itemize}
    \item Number of hidden layers: If the data is linearly separable, we do not need to use a hidden layer. Otherwise, we will use more hidden layers the more complex the separation is.
    \item Number of neurons in the respective layer
    \item Optimiser: we try different optimiser offered by the keras library\footnote{\url{https://keras.io/optimizers/}, accessed 03.01.2020}
\end{itemize}
As our output must be continuous, we can only use linear or ReLU activation function. The linear function can return negative values, while ReLU can only return values greater or equal to zero. As the predicted times are greater than zero, it makes no difference which of these activation functions we use. Besides, as per common knowledge \cite{Reed.1999}, we use mean-squared error as loss function.\\
We evaluate the final models with a separate test dataset and mean absolute error as metric.
\subsection{Android application}\label{sec:app_plans}
We implement the Android application "SwimPredictor" using Java, a minimum SDK version of 22 and a target SDK version of 28. The core functionality of the app is the usage of the trained models. Therefore, we implement an activity, where the coach can enter the information of the swimmer and the 50m time he measured in training and let the app predict the longer distance times based on this information. To make the app more appealing for coaches, we also include a stopwatch and a store functionality in the app. The coach can measure the 50m times directly in the app, predict the long-distance times and store the results and swimmer information to access it later when he makes the competition registration. The three functionalities can be accessed from a bottom tab navigation whats makes changing from one functionality to another very easy and intuitive.\\
The core or home fragment loads the pre-trained tensorflow-lite model it needs for the predictions from the app's assets folder. To use these models, we need to include the tensorflow-lite library in our app's \texttt{build.gradle} file. The coach can enter the information about the swimmer by choosing values from a dropdown list. This prevents the coach from entering unknown values or values in the wrong format. Moreover, the core fragment has a predict button, with which the user can trigger the prediction, two text views to display the predicted times and a store button to store the current record.\\
The stopwatch function implemented in the \texttt{org.swimpredictor.ui.clock} package has two buttons, a play/pause button and a reset button, as well as a text to display the measured time.\\
We implement the store functionality in the \texttt{org.swimprodictor.database} package using Android's Room \footnote{\url{https://developer.android.com/training/data-storage/room}, accessed 12.01.2020} and a SQLite database behind it. Although we only have one table, we consider a database the best way to store our data as it is rather easy to implement with Room and it assures consistency by built-in constraints. The database itself offers the functionality to insert and delete a sample. The database tab displays all samples stored in the database.