\section{Introduction}

%\begin{itemize}
%	\item Establish the scope of your report
%	\item Define the most important background information and the current state of the field your research is placed in
%	\item Describe your research approach, the problem you are trying to solve
%	\item Highlight potential outcomes that can be established as part of the report
%	\item Outline the structure of your paper
%\end{itemize}

% organisation into heats -> registration time important
In swimming competitions, every swimmer has his lane; thus, the competition is organised in heats. These heats are ordered by speed so the first heats are the slowest and the last are the fastest. To organise these heats, the coaches have to submit estimated times when registering their swimmer for the respective competition. A swimmer has a psychological benefit if he swims in a heat together with swimmers of the same speed; thus, it is essential that the estimated time is correct. For junior swimmers, however, it is hard to estimate the correct time as junior swimmers react strongly to stress and environmental influences\cite{mccarthy2013emotions} whats makes them more unreliable in competitions. Moreover, a new training input, e.g. an intense training camp, or growing up has a stronger impact for junior swimmers than for professionals thus, their performance changes more often. A possible solution for this could be to measure the times in training. For short distances this works very well, for longer distances, however, the training and competition times diverge. In this paper, we propose a mobile application that helps swim trainers to find the best registration times. The coaches can measure the 50m freestyle time in training and use the app to get the corresponding 100 and 200m freestyle times. To do so, we use a machine learning model that predicts the 100 and 200m time based on the 50m time as well as the age, gender and training age of the athlete. Besides, we investigate the relationship of different swimming distances and their times with a focus on junior swimmers.\\
This paper is structured as follows: In section \ref{sec:rel_work} we will explain the sports theory background and compare our work to current research. The results of this paper are explained in three separate parts. Each of the three parts explains its underlying approach as well as reports and discusses the respective results: In section \ref{sec:data_gen}, we focus on the new dataset we generate for this paper. Section \ref{sec:machine_learning} then explains and evaluates our machine learning approach. In section \ref{sec:app_plans} we  showcase the mobile application that uses the resulting machine learning model.